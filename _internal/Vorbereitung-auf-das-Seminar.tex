\documentclass{tufte-handout}

\usepackage[utf8]{inputenc}
\usepackage[autostyle=true,german=quotes]{csquotes}
\usepackage{hyphsubst} %\HyphSubstLet{ngerman}{ngerman-x-latest} % für die Trennung von zwei-schrittig
\usepackage[ngerman]{babel}   % neue deutsche Rechtschreibung (z. B. auch für Silbentrennung)

\title{Vorbereitung für das Seminar \enquote{Einführung ins Maschinelle Lernen: Hype oder Hybris}}

\author{Marvin Kastner \href{mailto:marvin.kastner@tuhh.de}{<marvin.kastner@tuhh.de>}}

%\date{28 March 2010} % without \date command, current date is supplied

\usepackage{graphicx} % allow embedded images
  \setkeys{Gin}{width=\linewidth,totalheight=\textheight,keepaspectratio}
  \graphicspath{{graphics/}} % set of paths to search for images

\usepackage{xurl}  % break url

\usepackage[
	type={CC},
	modifier={by},
	version={4.0},
	imagemodifier={-eu-88x31},
]{doclicense}


\begin{document}

\maketitle% this prints the handout title, author, and date

\begin{abstract}
\noindent
In diesem Dokument werden die Schritte aufgezählt, die von den Seminarteilnehmern selbständig \emph{vor} Beginn des Seminars auf dem eigenen Endgerät (PC, Laptop, Apple, ...) durchgeführt werden müssen.
Dies ist notwendig, weil zur Einrichtung der Arbeitsumgebung mehrere Gigabyte Daten heruntergeladen und verarbeitet werden.
Falls Sie während des Seminars keinen eigenen leistungsfähigen Laptop zur Verfügung haben sollten, sprechen Sie bitte frühzeitig die Seminarleitung an.
\end{abstract}

Im Folgenden wird aufgezählt, welche Software und welche Dateien benötigt werden.
Hier gibt es nur manchmal eine zeitliche Abhängigkeit, schauen Sie also gerne schon etwas voraus und parallelisieren Sie Aufgaben da, wo dies den Prozess beschleunigt.
Der zusätzliche Text beschreibt, warum diese Schritte gemacht werden und zeigt ein paar Alternativen auf.


\section{Installation von JupyterLab}

Gehen Sie auf
\url{https://www.anaconda.com/distribution/} 
und laden Sie die Anaconda-Version für Ihr Betriebssystem herunter.
Anaconda ist eine Python-Distribution, die vorkompilierte Bibliotheken ausliefert und damit den Aufwand beim Installieren von Bibliotheken minimiert.
Der Download-Bereich der Webseite sollte ungefähr wie in Abbildung\,\ref{fig:anaconda} aussehen.

\begin{marginfigure}
  \includegraphics{anaconda}
  \caption{Der Download-Bereich von Anaconda (Ausschnitt).}%
\label{fig:anaconda}
\end{marginfigure}

Falls Sie auf Ihrem Endgerät zwei Accounts, einen Administrator- bzw.\ root-Account und einen Account fürs alltägliche Arbeiten, verwenden, seien Sie bitte vorsichtig.
Die Installation von Anaconda erfordert \emph{keine} erhöhten Rechte.
Verwenden Sie bitte einfach den normalen Nutzeraccount.
Eine Installation durch einen Nutzer mit erhöhten Rechten  kann u.\,U. dazu führen, dass Anaconda nur für diesen  installiert ist und die Dateien für die anderen Nutzeraccount auf dem Endgerät nicht nutzbar sind.
Diese Probleme mit der Rechteverwaltung sind umfangreich und diese zu beheben ist sehr zeitaufwändig.

Rufen Sie nach dem Download den Installer auf und folgen Sie den Installationsschritten.
Konsultieren Sie im Fehlerfall offizielle Quellen des Herstellers (wie z.\,B. \url{https://docs.anaconda.com/anaconda/install/})
oder Foren (wie z.\,B. \url{https://stackoverflow.com}).


\newthought{Warum werden Jupyter Notebooks eingesetzt?}
Im Bereich Maschinelles Lernen und Data Science spielen Jupyter Notebooks eine immer größere Rolle.
Das Medienformat ist auf JSON-Basis und kann u.\,a.\
Text (d.\,h. Plaintext, Markdown, LaTeX und HTML),
Bilder und
ausführbaren Code enthalten.
So können die Arbeitsschritte für Dritte nachvollziehbar dokumentiert und die dazugehörigen Konzepte in einem Dokument zentral erklärt werden.


\section{Bezug der Seminar-Materialien}
\label{sec:git-clone}

Klonen Sie das git-Repository
\url{https://github.com/1kastner/machine-learning-hype-or-hybris},
damit Sie die Seminar-Materialien lokal haben -- dies umfasst nur den Programmierteil des Seminars.
Am einfachsten ist es, wenn Sie die Dateien lokal unterhalb des Ordners \texttt{Eigene Dateien} ablegen.
Denn in diesem Ordner öffnet sich standardmäßig JupyterLab.
Falls Sie noch nie mit git gearbeitet haben, lesen Sie bitte die nächsten Absätze.

\newthought{Warum sollte ich git lernen?}
Für die Versionsverwaltung ist git quasi der Standard und wird immer häufiger auch außerhalb der Software-Entwicklung, aus der git ursprünglich stammt, eingesetzt.
Deswegen lohnt es sich für (fast) jeden, sich Fähigkeiten mit diesem Tool anzueignen.
Das Original-Tool ist ein Kommandozeilentool, welches über \url{https://git-scm.com/} heruntergeladen werden kann.
Wer lieber grafische Oberflächen mag, kann sich eine von vielen GUI Clients%
\sidenote{Eine Liste ist auf \url{https://git-scm.com/download/gui/win} zu finden.}
aussuchen.
Hier sollte neben dem Betriebssystem auch die ggf.\ kostenpflichtige Lizenz beachtet werden.
Einige Lizenzen unterscheiden z.\,B. zwischen der privaten Verwendung und der Verwendung im Arbeitskontext.
Bis zum Start des Seminars werden u.\,U. die Materialien noch überarbeitet oder erweitert.
Aktualisieren Sie also bitte regelmäßig Ihre vorliegende Version über ein \texttt{git pull} bzw.\ durch das Klicken auf den Button \enquote{Pull} im GUI-Client Ihrer Wahl.

\newthought{Ist es für das Seminar zwingend notwendig, git zu lernen?}
Falls Ihnen git unbekannt ist und Sie keine Zeit dafür haben, sich mit git auseinanderzusetzen, gibt es auch die Möglichkeit, den Inhalt als ZIP-Ordner herunterzuladen.
Klicken Sie dafür auf den Button, wie er in Abbildung\,\ref{fig:github} zu sehen ist.
Falls Lernmaterialien später noch angepasst werden, müssen Sie diese dann allerdings erneut herunterladen und in einem neuen Ordner entpacken.

\begin{marginfigure}
  \includegraphics{github-zip}
  \caption{Ein GitHub-Repository bietet verschiedene Möglichkeiten zum Bezug der Inhalte an.}%
\label{fig:github}
\end{marginfigure}


\section{Installation der Bibliotheken}

Mit dem Anaconda Navigator können die für das Seminar  benötigten Bibliotheken automatisch inklusive der Abhängigkeiten installiert werden.
In der Abbildung\,\ref{fig:anaconda-navigator} sehen Sie unten den Button \enquote{Import} (im Screenshot mit einer (1) markiert).
Klicken Sie diesen an.
Damit öffnet sich das Fenster \enquote{Import new environment}.
Klicken Sie hier auf den Ordner in der Zeile, die mit 
\enquote{Specification File}
beginnt (mit einer (2) markiert).
Danach öffnet sich ein Fenster (mit einer (3) markiert), in dem Sie dann zu den von GitHub bezogenen Dateien navigieren können.
Wählen Sie die Datei \texttt{environment.yml} aus
-- sie liegt auf der obersten Ebene des Projektordners --
und klicken Sie auf \enquote{Öffnen}.
Das Erstellen der Umgebung nimmt für gewöhnlich mehrere Minuten in Anspruch.

\begin{figure}[h]
  \includegraphics{anaconda-navigator-import-new-environment--mit-reihenfolge}
  \caption{Der Anaconda Navigator erlaubt das Importieren von \texttt{environment.yml}-Dateien.}%
\label{fig:anaconda-navigator}
\end{figure}

\newthought{Warum ist dieser Schritt notwendig?}
Die Jupyter Notebooks, die Sie soeben heruntergeladen haben, benötigen eine bestimmte Umgebung, damit sie wie gewünscht geöffnet und ausgeführt werden können.
Zu dieser Umgebung gehört z.\,B. ein Python-Interpreter mit einer bestimmten Python-Version ebenso wie eine Reihe ausgewählter Python-Bibliotheken.
Auf der obersten Ebene des git-Repositorys befindet sich die Datei \texttt{environment.yml}, in der alle Anforderungen an die Entwicklungsumgebung aufgeführt werden.
Die Struktur der Datei \texttt{environment.yml} ist von Anaconda vorgegeben und erlaubt es, die Abhängigkeiten von Bibliotheken automatisch aufzulösen.
Damit man an einem Endgerät in verschiedenen Projekten unterschiedliche Versionen einer gleichen Bibliothek haben kann,
strukturiert Anaconda die zu einem Projekt gehörenden Bibliotheken standardmäßig in Umgebungen (eng. Environments).
Für das Seminar erstellen wir die Umgebung
\texttt{ml-hype-or-hybris}
basierend auf der gegebenen \texttt{environment.yml}.

\newthought{Geht es denn nur über die GUI?}
Natürlich gibt es auch ein Kommandozeilentool, das mit Anaconda ausgeliefert worden ist.
Es heißt \texttt{conda}.
Unter Windows wird dieses Tool je nach Auswahl während der Installation nicht in die Pfad-Variable aufgenommen.
Es steht Ihnen aber auf jeden Fall in der
\emph{Anaconda Powershell Prompt} (basierend auf der PowerShell)
zur Verfügung.
Auf
\url{https://docs.conda.io/projects/conda/en/latest/user-guide/tasks/manage-environments.html#creating-an-environment-from-an-environment-yml-file}
wird erläutert, wie eine existierende \texttt{environment.yml} eingelesen werden kann.
Folgen Sie der Anleitung und erstellen Sie die benötigte Umgebung.

\section{Start von JupyterLab}

JupyterLab kann nun über den Anaconda Navigator gestartet werden.
In Abbildung~\ref{fig:start-jupyterlab} ist dies abgebildet.
Zunächst wird links im Menü
(im Screenshot mit einer (1) markiert)
\enquote{Home} ausgewählt.
Im zweiten Schritt muss die Umgebung
\texttt{ml-hype-or-hybris}
ausgewählt werden
(mit einer (2) markiert).
Danach startet ein Klick auf \texttt{Launch} die Anwendung JupyterLab im Browser
(mit einer (3) markiert).
Standardmäßig öffnet sich nun ein neuer Tab im Browser und JupyterLab zeigt zunächst den Inhalt vom Ordner \texttt{Eigene Dateien} an.

\begin{figure}[h]
  \includegraphics{anaconda-navigator-jupyterlab--mit-reihenfolge}
  \caption{Aus dem Anaconda Navigator kann JupyterLab gleich in der richtigen Umgebung gestartet werden.}%
\label{fig:start-jupyterlab}
\end{figure}
%TODO Tausche Conda-Umgebung in Abbildung aus.

\section{Erste Schritte mit JupyterLab}

Wenn Sie die Seminar-Materialien wie unter Abschnitt \enquote{\nameref{sec:git-clone}} angegeben unterhalb des Ordners \texttt{Eigene Dateien} abgelegt haben, 
können Sie in JupyterLab zum Ordner 
\texttt{00-Installationscheck}
navigieren und dort das Jupyter Notebook (die Datei mit der Endung \texttt{.ipynb}) mit einem Doppelklick öffnen.

Überprüfen Sie, ob Sie alle Zellen ausführen können.
Falls es Fehlermeldungen gibt, melden Sie sich gerne vor Beginn de Veranstaltung per Mail bei der Seminarleitung.
Falls Sie noch keine Erfahrungen mit Python haben, schaueb Sie sich bitte als Vorbereitung auf den Workshop das Jupyter Notebook
\texttt{01-Einfuehrung-in-Python}
an.

\begin{figure}[h]
  \includegraphics{jupyterlab-running}
  \caption{In JupyterLab können links alle Dateien und Ordner aus dem Ordner \emph{Eigene Dateien} betrachtet werden.}%
\label{fig:start-jupyterlab}
\end{figure}

\doclicenseThis{}

\end{document}
